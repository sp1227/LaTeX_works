\documentclass[twoside,openright]{report}
\usepackage[hangul]{kotex}
\usepackage[parenthesis,hang]{dhucsfn}
\usepackage{tikz-cd}
\usepackage{adjustbox}
% \usepackage[backend=biber,style=apa,doi=false,isbn=false,url=false,natbib=true,citestyle=authoryear,bibstyle=authoryear]{biblatex}

% \addbibresource{phd.bib}
% \DefineBibliographyStrings{english}{andothers={외}} %글로벌하게 다 바꿔버리는 문제가 있다. 영어 문헌 3인 이상은 어쩌지

% \renewbibmacro*{textcite}{%
%     \ifnameundef{labelname}
%         {\iffieldundef{shorthand}
%         {\usebibmacro{cite:label}%
%             \setunit{%
%             \global\booltrue{cbx:parens}%
%             \printdelim{nonameyeardelim}\bibopenparen}%
%             \ifnumequal{\value{citecount}}{1}
%             {\usebibmacro{prenote}}
%             {}%
%             \usebibmacro{cite:labeldate+extradate}}
%         {\usebibmacro{cite:shorthand}}}
%         {\printnames{labelname}%
%         \setunit{%
%         \global\booltrue{cbx:parens}%
%         \bibopenparen}%
%         \ifnumequal{\value{citecount}}{1}
%         {\usebibmacro{prenote}}
%         {}%
%         \usebibmacro{citeyear}}}%

% \AtEveryCitekey{%
%   \renewcommand{\nameyeardelim}{ }%
%   \renewcommand{\postnotedelim}{:}%
%   \DeclareFieldFormat{postnote}{#1}%
%   \ifkeyword{kobib}{%
%     \renewcommand{\multinamedelim}{·}%
%     \renewcommand{\finalnamedelim}{·}%
%     }{%
%   }%
% }
% \AtEveryBibitem{%
%   \ifkeyword{kobib}{%
%     \renewcommand{\multinamedelim}{·}%
%     \renewcommand{\finalnamedelim}{·}%
%     \renewbibmacro{in:}{}
%     \DeclareFieldFormat{journaltitle}{ 《#1》}%
%     \DeclareFieldFormat{booktitle}{ 《#1》}%
%     \DeclareFieldFormat{pages}{#1}%
%     }{%
%   }%
% }

\usepackage{geometry}
\usepackage{lineno}
\usepackage{booktabs}
\usepackage{enumitem}
\usepackage{titlesec}
% \usepackage{hanging}
\usepackage[utf8]{inputenc}
\usepackage{indentfirst}
\usepackage{array,longtable}
\usepackage{lipsum}
\usepackage{multicol}
\usepackage{multirow}
\usepackage{tipa}
% \usepackage{pdfpages}
\usepackage{graphicx}
\graphicspath{ {./images/} }
\usepackage{stmaryrd}
% \usepackage{amssymb}
\usepackage{hyperref}
\usepackage{gb4e}   %gb4e must be at the end of the usepackage commands
\counterwithin{exx}{chapter}
\makeatletter
\def\@@xsii{\let\@xsii\jaso}
\makeatother
\geometry{
    % 심사 후 인쇄 넘기기 전에 A4 크기로 용지 크기와 여백을 조정할 것
    paperheight=297mm, % dissertation 190 * 260, A4 210 * 297, diff 20 * 37 (17 + 19)
    paperwidth=210mm,
    inner=30mm,         % Inner margin A4 38mm
    outer=30mm,         % Outer margin A4 38mm
    bindingoffset=4mm, % Binding offset
    top=28mm,           % Top margin A4 60mm
    bottom=28mm,        % Bottom margin A4 60mm
    % showframe,         % show how the type block is set on the page
}
\defaultfontfeatures{Mapping=tex-text,Ligatures=TeX}
\setmainfont[Ligatures=TeX]{Times New Roman} 
\setmainhangulfont[Script=Hangul]{Source Han Serif K}
\setsanshangulfont[Script=Hangul]{Source Han Sans K}
\linespread{1.7}
% \setlength{\footnotesep}{1.3em}
% \setlength{\parindent}{1em}
% \setlength{\parskip}{0em}
% \setcounter{secnumdepth}{3}
% \setcounter{tocdepth}{3}
% \setlist[itemize]{noitemsep}
% \setlist[enumerate]{noitemsep}
% \setlist[description]{noitemsep}
% \titleformat{\chapter}[hang]{\normalfont\LARGE}{제\thechapter 장}{10pt}{}[]
% \titlespacing{\chapter}{0pt}{-50pt}{30pt}
% \titleformat{\section}[hang]{\normalfont\large}{\thesection}{10pt}{}[]
% \titleformat{\subsection}[hang]{\normalfont}{\thesubsection}{10pt}{}[]
% \titleformat{\subsubsection}[hang]{\normalfont}{\thesubsubsection}{10pt}{}[]
% \titleformat{\paragraph}[hang]{\normalfont}{\theparagraph}{10pt}{}
\newcommand{\sa}[1]{〈#1〉}
\newcommand{\da}[1]{{\small{《#1》}}}
\newcommand{\sq}[1]{‘#1'}
\newcommand{\dq}[1]{“#1"}
% \newcommand{\tita}[1]{\underline{#1}}
% \newcommand{\num}[1]{\textsuperscript{#1}}
% \newcommand{\marginalnote}[1]{{\small{(#1)}}}

\begin{document}


% --- Title Information ---
\title{2025-2 문헌학연습 분석 논문 목록}
\author{김미경}
\date{\today}

% \maketitle % Prints the title


\begin{center}
    {\Large\bfseries 2025-2 문헌학연습 분석 논문 목록\par} % Title
    \vspace{1.5em} % Space after title
    {\large 김미경\par} % Author
    \vspace{0.5em} % Space after author
    {\normalsize \today\par} % Date
\end{center}
\vspace{2em} % Space between title block and the main content

\section*{안내}
\begin{itemize}
  \item 2-5주, 7주 총 5주간 하루에 패키지 2개씩 다룸
  \item 패키지 1개당 발표자 1인
  \item 자신이 고른 패키지 안에서 교체하고 싶은 논문이 있으면 상의 후 교체 가능
  \item 발표 순서는 패키지 번호와 무관함
\end{itemize}
\section*{분석 대상}

\begin{itemize}
  \item 할 일
    \begin{itemize}
      \item 논문에서 수행한 연구의 연구질문은 무엇인가
      \item 이 연구질문에 답하기 위해 어떤 데이터를 선정하였는가
      \item (데이터를 직접 만들었다면 어떤 점에 주의하여 만들었는가)
      \item 데이터에서 어떤 표현을 어떤 방법으로 추출하였는가
      \item 이 연구의 발견을 재현하려 한다면 나는 어떤 데이터를 어떻게 사용하겠는가
    \end{itemize}
  \item 하지 않을 일
    \begin{itemize}
      \item 논문의 주장은 무엇인가
      \item 논문의 이론적 쟁점은 무엇인가
      \item 논문의 논리 구조는 어떻게 되어 있는가
      \item 논문의 근거는 합당한가
    \end{itemize}
\end{itemize}

\section*{패키지 1: 비규범적 언어 사용 양상 분석}
\begin{enumerate}[label=\textbf{\arabic*.}] % Makes the number bold with a period
  
  \item \textbf{이래호 (2012).} 선어말 어미 \sq{-시-}의 청자 존대 기능에 대한 고찰, \da{언어학 연구}, 23, 147-166. \href{https://doi.org/10.17002/sil..23.201204.147}{https://doi.org/10.17002/sil..23.201204.147}

        \begin{description}[font=\normalfont\bfseries, style=unboxed]
          \item[연구질문] 청자 존대법 \sq{-시-}는 표준적인 주체 존대법 '-시-'와 어떻게 다르며, 이 형태는 어떤 의미적, 화용론적 기능을 수행하는가?
            \begin{itemize}
              \item 청자 존대법 \sq{-시-}가 문장의 서술어와 결합하기 위해 충족되어야 하는 의미적 조건은 무엇인가?
              \item 화자가 청자 존대법 \sq{-시-}를 반복적으로 사용하는 것은 어떤 목적을 달성하기 위한 행위인가?
            \end{itemize}
          \item[데이터 요약] 일화적 증거(Anecdotal Evidence)
        \end{description}

  \item \textbf{이정복 (2022).} 누리꾼들의 상황 주체 높임 \sq{-시-} 사용 연구, \da{방언학}, 35, 107-144. \href{https://kiss-kstudy-com-ssl.libproxy.snu.ac.kr/Detail/Ar?key=3957996}{\u{KISS 링크}}

  \begin{description}[font=\normalfont\bfseries, style=unboxed]
    \item[연구질문] 한 외제 자동차 동호회 온라인 카페에서, 누리꾼들은 상황 주체 높임 \sq{-시-}를 어떻게 사용하고 있으며, 이러한 사용 양상은 한국어 경어법의 변화와 확산에 대해 무엇을 보여주는가?
      \begin{itemize}
        \item 이 카페에서 상황 주체 높임 \sq{-시-}는 어떤 서술어 유형과 함께, 누구를 대상으로 사용되는가?
        \item 누리꾼들이 상황 주체 높임 \sq{-시-}를 사용하는 사회적 동기는 무엇이며, 이 표현은 커뮤니티 내에서 어떤 기능을 수행하는가?
        \item 이러한 \sq{-시-}의 사용은 커뮤니티 구성원들 사이에서 얼마나 널리, 그리고 자연스럽게 받아들여지고 있는가?
      \end{itemize}
    \item[데이터 요약] 2021년 1월 \textasciitilde 3월간 사용된 외제 자동차 동호회 인터넷 카페의 게시글
  \end{description}

  \item \textbf{김은혜 (2016).} 한국어 선어말 어미 \sq{-시-}의 사물 존대 기능 : 백화점, 대형마트, 재래시장 판매원의 발화를 중심으로, \da{사회언어학}, 24(1), 91-113. \href{http://dx.doi.org/10.14353/sjk.2016.24.1.04}{http://dx.doi.org/10.14353/sjk.2016.24.1.04}

  \begin{description}[font=\normalfont\bfseries, style=unboxed]
    \item[연구질문] 기존 주체 높임법과 달리, 사물에까지 사용되는 선어말어미 ‘-시-’의 새로운 기능은 무엇이며, 이러한 언어 변이는 사회 계층과 어떤 관련이 있는가?
      \begin{itemize}
        \item 고급 서비스 산업(백화점)과 일반 상업 공간(재래시장)에서, ‘-시-’의 사물 존대 용법은 어떻게 다르게 나타나는가?
        \item 이러한 ‘-시-’의 새로운 용법에 대한 한국어 화자들의 인식은 어떠한가?
        \item 이러한 실제 언어 사용의 변화를 한국어 교육에서 어떻게 다루어야 하는가?
      \end{itemize}
    \item[데이터 요약] 백화점/대형마트/재래시장 점원과 연구자의 대화, 설문조사
  \end{description}

\end{enumerate}

\section*{패키지 2: 새로 나타난 언어 사용 양상 분석}

\begin{enumerate}[label=\textbf{\arabic*.}] % Makes the number bold with a period

  % --- Add New Papers Here ---
  \item \textbf{이정민 (2023).} \sq{줄 모르다/알다} 구문 분석 - 카카오톡 메신저 말뭉치를 중심으로 -, \da{우리말글}, 98, 85-123. \href{https://doi.org/10.18628/urimal.98..202309.85}{https://doi.org/10.18628/urimal.98..202309.85}
  
        \begin{description}[font=\normalfont\bfseries, style=unboxed]
          \item[연구질문] 카카오톡 메신저 말뭉치에서 나타나는 \sq{줄 모르다/알다} 구문의 통사·의미적 사용 양상은 어떠하며, 이 구문에서 서술어 생략 현상은 어떤 조건에서, 어떤 화용론적 기능을 위해 발생하는가?
            \begin{itemize}
              \item \sq{줄 모르다/알다} 구문은 \sq{방법/능력}과 \sq{사실 인식}이라는 두 가지 의미 중 어떤 의미로 더 자주 사용되는가?
              \item 두 의미 용법은 어떤 형태·통사적 제약의 차이를 보이는가?
              \item 각 구문이 가장 자주 실현되는 의미 유형과 결합하는 문법 형태는 무엇인가?
              \item \sq{모르다}와 \sq{알다}의 생략에 영향을 미치는 요소는 무엇인가?
              \item 서술어가 생략된 형태(\sq{줄은}, \sq{줄이야} 등)는 원래 구문의 의미를 유지하면서 어떤 추가적인 기능을 수행하는가?
            \end{itemize}
          \item[데이터 요약] 국립국어원 어휘 의미 분석 말뭉치 2020(버전 2.0), 국립국어원 메신저 말뭉치(버전 2.0)
        \end{description}

  \item \textbf{윤재연 (2020).} \sa{때문에}의 비문법적 사용에 대한 연구, \da{사회언어학}, 28(3), 245-277. \href{https://scholar-kyobobook-co-kr-ssl.libproxy.snu.ac.kr/article/detail/4010027923589}{교보 스콜라 링크}
  
        \begin{description}[font=\normalfont\bfseries, style=unboxed]
          \item[연구질문] \sq{그래서}와 같이 접속사처럼 사용되는 \sq{때문에}의 비문법적 용법은, 선행 연구의 주장처럼 탈문법화(degrammaticalization) 현상으로 보아야 하는가, 아니면 교정해야 할 단순한 문법적 오류로 보아야 하는가?
            \begin{itemize}
              \item 접속사적 \sq{때문에}는 어떤 언어 환경에서 나타나는가?
              \item 이 현상은 특정 문법적 변화로 인정될 만큼 충분히 빈번하게 나타나는가?
              \item 만약 이 현상이 의존명사 '때문'의 자립성 때문이라면, 이 자립성은 '때문'이 사용되는 다른 문법 구조에서도 일관되게 나타나는가?
            \end{itemize}
          \item[데이터 요약] 고려대학교 민족문화연구원 SJ-RIKS, SJ-RIKS ext (주로 1990년대 자료로 추정) / 고려대학교 민족문화연구원 Trends21 코퍼스(주로 2000~2013년 자료) / 국립국어원 말뭉치 
            \begin{itemize}
              \item 발표자 주의: 이 연구에서 사용한 국립국어원 말뭉치가 구체적으로 무슨 자료일지 추정할 것
            \end{itemize}
        \end{description}

  \item \textbf{하영우, 신지영 (2012).} 자유 발화 자료에 나타난 \sq{-어가지고}의 문법 범주와 의미 기능, \da{한국어학}, 55, 397-430. \href{https://www.dbpia.co.kr/journal/articleDetail?nodeId=NODE06568902}{DBPIA 링크}
  
        \begin{description}[font=\normalfont\bfseries, style=unboxed]
          \item[연구질문] 구어에서 나타나는 연결어미 \sq{-어가지고}의 문법적 지위와 의미 기능은 무엇이며, 이는 \sq{-어서}와 어떻게 다른가?
            \begin{itemize}
              \item 연결어미 \sq{-어가지고}는 문법화 과정을 통해 완전한 어미의 지위를 획득했는가? 만약 그렇지 않다면, 어떤 제약을 보이는가?
              \item 연결어미 \sq{-어가지고}는 어떤 의미 기능들로 사용되는가?
              \item 의미 기능의 사용 빈도 분포 측면에서, \sq{-어가지고}는 \sq{-어서}와 어떤 차이점을 보이는가?
            \end{itemize}
          \item[데이터 요약] 고려대학교 음성언어정보연구실 자유발화 자료(23시간 3분 35초, 172,084 어절)
        \end{description}

\end{enumerate}

\section*{패키지 3: 연관 단어 분석}

\begin{enumerate}[label=\textbf{\arabic*.}] % Makes the number bold with a period

  % --- Add New Papers Here ---
  \item \textbf{김해연 (2019).} 반의어 \sq{묶다}와 \sq{풀다}의 의미와 용법에 대한 코퍼스언어학적 분석, \da{담화와 인지}, 26(3), \href{https://doi.org/10.15718/discog.2019.26.3.43}{https://doi.org/10.15718/discog.2019.26.3.43}
  
        \begin{description}[font=\normalfont\bfseries, style=unboxed]
          \item[연구질문] 한국어 반의어 \sq{묶다}와 \sq{풀다}의 실제 사용 양상은 어떻게 다르며, 이를 통해 두 어휘의 반의 관계가 왜 비대칭적이라고 할 수 있는가?
            \begin{itemize}
              \item 두 어휘가 함께 사용되는 명사의 유형은 어떻게 다르며, 이 차이는 \sq{풀다}의 의미가 더 다양하게 확장되었음을 어떻게 보여주는가?
              \item 개념 은유 이론의 관점에서 볼 때, 두 어휘의 비유적 의미는 어떤 은유적 사고를 통해 파생되는가?
              \item 두 어휘의 사용 빈도, 공기 관계, 연어 패턴의 비대칭성은 이 둘의 관계가 완전한 반의 관계가 아닌 부분적 반의 관계(partial antonymy)임을 어떻게 증명하는가?
            \end{itemize}
          \item[데이터 요약] 세종코퍼스 > 현대문어 > 책 > 상상적텍스트-일반
            \begin{itemize}
              \item 발표자 주의: 이 연구에서 사용한 세종코퍼스가 구체적으로 무슨 자료일지 추정할 것
            \end{itemize}
        \end{description}

  \item \textbf{신중진 (2018).} [다름]의 \sq{틀리다}를 형성하는 유의-반의 관계망 분석, \da{한국어학}, 78, 31-54. \href{https://doi.org/10.20405/kl.2018.02.78.31}{https://doi.org/10.20405/kl.2018.02.78.31}
  
        \begin{description}[font=\normalfont\bfseries, style=unboxed]
          \item[연구질문] \sq{틀리다}가 [다름]의 의미를 획득하게 된 과정은 무엇이며, 이 의미 확대 과정에서 유의 관계와 반의 관계에 기반한 유추는 각각 어떤 역할을 하는가?
            \begin{itemize}
              \item \sq{틀리다↔맞다}라는 기존의 반의 관계는 이 의미 변화의 출발점으로서 어떻게 기능하는가?
              \item \sq{맞다=같다}라는 유의 관계는 \sq{틀리다}와 \sq{다르다}를 연결하는 의미망 속에서 어떤 다리 역할을 하는가?
              \item \sq{같다/맞다↔다르다}라는 반의 관계를 기반으로 한 반의어적 유추가, 최종적으로 \sq{틀리다=다르다}라는 새로운 관계를 형성하는 과정은 구체적으로 어떠한가?
            \end{itemize}
          \item[데이터 요약] 21세기 세종말뭉치
            \begin{itemize}
              \item 발표자 주의: 이 연구에서 사용한 21세기 세종말뭉치가 구체적으로 무슨 자료일지 추정할 것. 단서는 ‘어의별 용례’와 검색 개수 ‘800여 개’
            \end{itemize}
        \end{description}
        
  \item \textbf{최지희 (2020).} 한국어 학술 정형 표현 연구: 빈도와 핵심도를 중심으로, \da{언어와 정보 사회}, 40, 113-147, \href{https://doi.org/10.29211/soli.2020.40..005}{https://doi.org/10.29211/soli.2020.40..005}
  
        \begin{description}[font=\normalfont\bfseries, style=unboxed]
          \item[연구질문] 한국어 학술 논문에서 통계적으로 유의미하게 자주 사용되는 핵심적인 관용적 표현(academic formulaic expressions)은 무엇이며, 그 사용 양상은 학문 분야별로 어떻게 다른가?
            \begin{itemize}
              \item 대규모 한국어 학술 논문 말뭉치에서, 형태소 기반 N-gram 분석을 통해 추출할 수 있는 유의미한 학술적 관용 표현은 무엇인가?
              \item 추출된 표현들 중, 빈도와 통계적 유의성을 기준으로 어떤 것들이 핵심 학술 관용 표현으로 분류될 수 있는가?
              \item 이러한 핵심 학술 관용 표현들은 인문학, 사회과학, 예술, 자연과학, 공학 등 5개 학문 분야에서 공통적으로 나타나는가, 아니면 분야별로 특화된 사용 양상을 보이는가?
            \end{itemize}
          \item[데이터 요약] 학술지 논문을 수집하여 만든 ‘학술지 말뭉치’
        \end{description}
        
\end{enumerate}

\section*{패키지 4: 품사 기반 결합 양상 분석}

\begin{enumerate}[label=\textbf{\arabic*.}] % Makes the number bold with a period

  % --- Add New Papers Here ---
  \item \textbf{정성훈 (2024).} 한국어 용언의 다층 네트워크 분석 - 한국어 부사와 용언을 중심으로 -, \da{한국어학}, 105, 121-153, \href{http://dx.doi.org/10.20405/kl.2024.11.105.121}{http://dx.doi.org/10.20405/kl.2024.11.105.121}
  
        \begin{description}[font=\normalfont\bfseries, style=unboxed]
          \item[연구질문] '어휘의미분석 말뭉치'에 나타난 부사와 용언의 수식 관계를 바탕으로, 한국어 용언 네트워크를 어떻게 구축하고 분석할 수 있으며, 이 네트워크는 구어/문어, 동사/형용사와 같은 여러 언어 층위에서 한국어 용언 체계의 구조적 특징을 어떻게 보여주는가?
            \begin{itemize}
              \item 말뭉치에서 부사와 용언 사이의 유의미한 수식 관계는 무엇이며, 이를 어떻게 추출할 수 있는가?
              \item 추출된 관계를 바탕으로 부사와 용언의 2원 네트워크(bipartite network)를 어떻게 구축할 수 있는가?
              \item 이 네트워크와 네트워크 투사(network projection)를 통해 드러나는 한국어 용언 간의 관계는 무엇이며, 이는 용언의 의미적 속성을 어떻게 설명하는가?
              \item 구어와 문어, 동사와 형용사 등 다층 네트워크 모델링을 적용했을 때, 한국어 용언의 사용 양상과 그 특징은 어떻게 다르게 나타나는가?
                \begin{itemize}
                  \item 발표자 주의: 이 연구에서 \sq{2원 네트워크}로 어떤 언어 현상을 설명하고 있는지 쉽게 설명할 것
                  \item 발표자 주의: 이 연구에서 \sq{네트워크 투사(network projection)}로 어떤 언어 현상을 설명하고 있는지 쉽게 설명할 것
                \end{itemize}
            \end{itemize}
          \item[데이터 요약] 국립국어원 어휘 의미 분석 말뭉치(버전 2.0)
        \end{description}

  \item \textbf{최운호 (2015).} 한국어 \sq{용언 어간 + 어미} 결합의 양상 —용언별 결합 어미 분포의 집중화 경향에 대하여—, \da{언어학}, 71, 17-47, \href{http://dx.doi.org/10.17290/jlsk.2015..71.17}{http://dx.doi.org/10.17290/jlsk.2015..71.17}
  
        \begin{description}[font=\normalfont\bfseries, style=unboxed]
          \item[연구질문] 한국어 용언들은 활용 형태의 다양성 측면에서 어떻게 다르며, 이 다양성을 어떻게 계량적으로 측정하고 설명할 수 있는가?
            \begin{itemize}
              \item 각 용언이 보이는 활용형의 다양성을 어떻게 객관적인 수치로 측정하여 척도화할 수 있는가?
              \item 측정된 다양성 척도를 어떻게 시각화하여, '모자란 움직씨'부터 기능 동사 '하다'에 이르기까지 용언들이 보이는 다양한 스펙트럼을 효과적으로 제시할 수 있는가?
              \item 다양성의 다른 측면인 '집중화' 관점에서, 각 용언은 특정 활용형에 얼마나 편중되어 사용되는 경향을 보이는가?
              \item 이러한 계량적 분석을 통해, 개별 용언들이 활용 양상에서 보이는 질적인 차이를 어떻게 증명할 수 있는가?
            \end{itemize}
          \item[데이터 요약] 21세기 세종계획 최종 성과물(2011. 12. 수정판) > 형태ᆞ의미분석 말뭉치
        \end{description}

  \item \textbf{최운호 (2017).} 한국어 ‘명사 + 격조사’ 결합의 양상, \da{언어학}, 78, 3-30, \href{http://dx.doi.org/10.17290/jlsk.2017..78.3}{http://dx.doi.org/10.17290/jlsk.2017..78.3}
  
        \begin{description}[font=\normalfont\bfseries, style=unboxed]
          \item[연구질문] 실제 말뭉치 사용에 기반했을 때, 한국어 일반 명사는 격조사와 어떤 결합 양상과 분포를 보이며, 이를 통해 기존의 불완전 곡용 논의를 어떻게 실증적으로 검증하고 확장할 수 있는가?
            \begin{itemize}
              \item 말뭉치에서 개별 명사(예: '등골')는 어떤 격조사들과 결합하며, 각 결합 형태의 빈도 분포는 어떠한가?
              \item 특정 명사의 전체 사용에서, 특정 격조사 결합형이 차지하는 비중(집중도)은 얼마나 되는가?
              \item 개별 명사들의 분석 결과를 종합했을 때, 한국어 명사 전반에 걸쳐 나타나는 격조사 결합의 보편적인 경향성은 무엇인가?
                \begin{itemize}
                  \item 발표자 주의: 최운호(2015)와 비교하여 공통점과, 새로이 시도된 점을 설명할 것
                \end{itemize}
            \end{itemize}
          \item[데이터 요약] 21세기 세종계획 최종 성과물(2011. 12. 수정판) > 형태ᆞ의미분석 말뭉치
        \end{description}
\end{enumerate}

\section*{패키지 5: 구어 특유의 문법 현상 분석}

\begin{enumerate}[label=\textbf{\arabic*.}] % Makes the number bold with a period

  % --- Add New Papers Here ---
  \item \textbf{서은아, 남길임, 서상규 (2004).} 구어 말뭉치에 나타난 조각문 유형 연구, \da{한글}, 264, 123-151. \href{https://www.kci.go.kr/kciportal/ci/sereArticleSearch/ciSereArtiView.kci?sereArticleSearchBean.artiId=ART000936604}{KCI 링크}
  
        \begin{description}[font=\normalfont\bfseries, style=unboxed]
          \item[연구질문] 대학생 구어 말뭉치에서 나타나는 조각문(fragment sentences)의 통사적 유형은 무엇이며, 그 특징과 실현 양상은 어떠한가?
            \begin{itemize}
              \item 문어 중심의 문장 개념이 아닌 억양 단위를 기준으로 분석했을 때, 구어 말뭉치에서 발견되는 조각문의 통사적 유형에는 어떤 것들이 있는가?
              \item 가장 빈번하게 나타나는 조각문 유형은 본래의 통사적 기능을 넘어, 대화 속에서 어떤 추가적인 화용론적 기능을 수행하는가?
              \item 구어 말뭉치에 통사 주석 정보를 부가하고 분석하는 것은, 기존의 문어 중심 문법이 설명하지 못했던 구어의 고유한 문장 유형과 특징을 밝히는 데 어떻게 기여하는가?
            \end{itemize}
          \item[데이터 요약] 연세대학교 언어정보연구원 구어 말뭉치 2만 어절(총 문장 4,760 개)
        \end{description}

  \item \textbf{이동석 (2014).} \sq{-냐}계 어미의 결합 분포에 대하여 -구어 말뭉치 분석을 중심으로-, \da{민족문화연구}, 64, 247-281. \href{http://dx.doi.org/10.17948/kcs.2014..64.247}{http://dx.doi.org/10.17948/kcs.2014..64.247}
  
        \begin{description}[font=\normalfont\bfseries, style=unboxed]
          \item[연구질문] '-냐'계 의문형 어미('-냐', '-으냐', '-느냐')의 실제 구어 사용 양상은 기존 문법 기술과 어떻게 다르며, 이러한 차이를 바탕으로 이 어미들의 결합 제약을 어떻게 새롭게 기술해야 하는가?
            \begin{itemize}
              \item 문법서에서 설명하는 '-으냐'와 '-느냐'의 표준적인 결합 규칙은 실제 구어 말뭉치에서 얼마나 낮은 비율로 나타나는가?
              \item 실제 구어에서 가장 지배적으로 나타나는 결합 패턴은 무엇인가? 
              \item 문법적으로 설명되지만 실제로는 드물게 사용되는 '-느냐'는 어떤 특수한 환경에 한정되어 나타나는가?
              \item 이러한 말뭉치 분석 결과를 바탕으로 국어사전과 문법서의 기술 내용은 현재 언어 현실을 반영하기 위해 어떻게 수정되어야 하는가?
            \end{itemize}
          \item[데이터 요약] 21세기 세종 계획 구어 말뭉치
            \begin{itemize}
              \item 발표자 주의: 이 연구에 사용된 21세기 세종 계획 구어 말뭉치가 원시 말뭉치인지 형태 분석 말뭉치인지 조사할 것
            \end{itemize}
        \end{description}

  \item \textbf{이은경 (2015).} 구어 텍스트에서의 목적격 조사의 비실현 양상, \da{우리말글}, 64, 57-86. \href{http://dx.doi.org/10.18628/urimal.64..201503.57}{http://dx.doi.org/10.18628/urimal.64..201503.57}
  
        \begin{description}[font=\normalfont\bfseries, style=unboxed]
          \item[연구질문] 한국어 구어 텍스트에서 목적격 조사의 비실현(생략)에 영향을 미치는 언어학적 요인들은 무엇이며, 각 요인은 어떤 조건에서 비실현을 유도하는가?
            \begin{itemize}
              \item 선행 체언의 유형은 목적격 조사의 비실현에 어떤 영향을 미치는가?
              \item 후행 서술어의 종류 및 선행 명사구와의 거리는 목적격 조사의 비실현에 어떤 영향을 미치는가?
              \item 후행 어미의 유형은 목적격 조사의 비실현에 어떤 영향을 미치는가?
              \item 담화의 특성은 목적격 조사의 비실현과 어떤 관련이 있는가?
            \end{itemize}
          \item[데이터 요약] 21세기 세종계획 최종 성과물 DVD에 수록된 현대 구어 형태 분석 말뭉치 내 4개 파일
        \end{description}

\end{enumerate}

\section*{패키지 6: 사용량 분석}

\begin{enumerate}[label=\textbf{\arabic*.}] % Makes the number bold with a period

  % --- Add New Papers Here ---
  \item \textbf{문재현, 김미란 (2018).} 탈어휘 동사 get의 구문과 의미 유형별 빈도: 교과서 듣기지문과 구어자료 비교, \da{언어와 정보 사회}, 34, 59-86, \href{http://dx.doi.org/10.29211/soli.2018.34..003}{http://dx.doi.org/10.29211/soli.2018.34..003}
  
        \begin{description}[font=\normalfont\bfseries, style=unboxed]
          \item[연구질문] 고등학교 영어 교과서에 나타난 탈어휘 동사 \sq{get}의 사용 빈도 및 양상은 실제 구어 데이터와 어떻게 다른가?
            \begin{itemize}
              \item 문법적 구조와 의미 유형의 분포 측면에서, 교과서의 'get' 사용은 실제 구어와 비교했을 때 어떤 과용(overuse) 또는 과소사용(underuse) 경향을 보이는가?
              \item 교과서와 실제 언어 사용 사이의 이러한 불일치는, 영어 학습자의 빈도 기반 어휘 학습과 자연스러운 어휘 선택에 어떤 잠재적 문제를 야기할 수 있는가?
            \end{itemize}
          \item[데이터 요약] 고등학교 영어 교과서 8종의 듣기 활동 지문, SBC(Santa Barbara Corpus of Spoken American English), 온라인 테드(TED) 강연 강연 전사(transcripts) 423개
        \end{description}

  \item \textbf{이나, 이찬규 (2025).} 신어의 사용 빈도와 사전 등재의 관련성에 관한 연구 - 신어 정착 요인의 정량적 기준 설정 -, \da{언어학 연구}, 74, 61-91. \href{http://dx.doi.org/10.17002/sil..74.202501.61}{http://dx.doi.org/10.17002/sil..74.202501.61}
  
        \begin{description}[font=\normalfont\bfseries, style=unboxed]
          \item[연구질문] 신조어의 정착에 영향을 미치는 요인들을 통계적으로 측정할 수 있는 정량적 기준은 무엇이며, 이 기준은 신조어의 사전 등재 가능성을 얼마나 잘 예측하는가?
            \begin{itemize}
              \item 빅카인즈(BIG KINDS)와 같은 실제 미디어 사용 데이터를 활용하여 신조어의 실제 사용 빈도, 등장 기사 수, 연도별 분포를 어떻게 측정할 수 있는가?
              \item 이러한 측정치를 바탕으로 신조어 정착을 판단하는 객관적이고 정량적인 기준을 어떻게 설정할 수 있는가?
              \item 제안된 정량적 기준을 충족하는 신조어들은 실제로 전자 사전에 등재될 확률과 어떤 상관관계를 보이는가?
            \end{itemize}
          \item[데이터 요약] 국립국어원 2012년 신어 자료집, 빅카인즈(BIG KINDS) 2012년 1월 1일 ~ 2023년 12월 31일 기사
        \end{description}

  \item \textbf{정지은 (2019).} 한국어 사전 표제어 발음형의 음소 빈도 연구, \da{언어와 언어학}, 83, 179-218. \href{http://dx.doi.org/10.20865/20198307}{http://dx.doi.org/10.20865/20198307}
  
        \begin{description}[font=\normalfont\bfseries, style=unboxed]
          \item[연구질문] 『우리말샘』 표제어의 발음 정보에 나타난 한국어 음소의 빈도 분포는 어떠한가?
            \begin{itemize}
              \item 음절을 초성, 중성, 종성으로 나누었을 때, 각각의 음소별 사용 빈도 순위는 어떻게 나타나는가?
              \item 초성, 중성, 종성의 빈도는 어두, 어중, 어말이라는 음절 위치에 따라 어떻게 다르게 나타나는가?
              \item 이러한 빈도 분석 결과를 바탕으로, 각 음절 위치(어두, 어중, 어말)에서 가장 대표적으로 사용되는 음절은 무엇인가?
            \end{itemize}
          \item[데이터 요약] 우리말샘 표제어
            \begin{itemize}
              \item 발표자 주의: 데이터의 구체적인 선정 기준과, 데이터 선정에 대한 심사자 의견 및 그에 대한 답변도 논의에 포함시킬 것
            \end{itemize}
        \end{description}


\end{enumerate}

\section*{패키지 7: 장르별 차이 분석}

\begin{enumerate}[label=\textbf{\arabic*.}] % Makes the number bold with a period

  % --- Add New Papers Here ---
  \item \textbf{김령환 (2020).} \sq{때문}의 텍스트 장르별 사용 양상과 의미 특성 -구어, 신문, 소설, 학술논문 텍스트를 중심으로-, \da{語文學}, 148, 3-32. \href{https://doi.org/10.37967/emh.2020.06.148.3}{https://doi.org/10.37967/emh.2020.06.148.3}
  
        \begin{description}[font=\normalfont\bfseries, style=unboxed]
          \item[연구질문] 구어, 소설, 신문, 학술논문이라는 네 가지 장르에 따라, '때문'의 사용 양상과 의미 특성은 어떻게 다르게 나타나는가?
            \begin{itemize}
              \item 네 가지 장르에 따라 ‘때문에’, ‘때문이다’ 등 주요 활용형의 사용 빈도 분포는 어떻게 다른가?
              \item 각 장르에서 특징적으로 나타나는 의미적, 화용론적 기능은 무엇인가?
            \end{itemize}
          \item[데이터 요약] 21세기 세종계획 형태 분석 (현대 문어) 말뭉치, 21세기 세종계획 현대 구어 말뭉치의 일상대화, 국어 교과의 <문학> 과목의 현대 소설 작품 58편, 21세기 세종계획 원시 문어 말뭉치 중 신문, 국어국문학 분야의 전문 학술지 \da{어문론총} 36~54호에 실린 논문 138편, 
        \end{description}
        
  \item \textbf{이유라 (2021).} 한국어 정도부사의 사용역별 출현 빈도 및 헤지성 분석, \da{언어학 연구}, 60, 115-131. \href{http://doi.org/10.17002/sil..60.202107.115}{http://doi.org/10.17002/sil..60.202107.115}
  
        \begin{description}[font=\normalfont\bfseries, style=unboxed]
          \item[연구질문] 한국어 정도부사들은 사용역(register)에 따라 어떤 빈도 분포와 사용 양상의 차이를 보이며, 이러한 차이는 각 부사가 가진 '헤지(hedge)' 기능과 어떤 관련이 있는가?
            \begin{itemize}
              \item 구어, 소설, 신문, 학술논문이라는 네 가지 사용역에 따라, 각 정도부사의 출현 빈도는 어떻게 다르게 나타나는가?
              \item 각 정도부사의 사용 양상은 특정 사용역(예: 구어, 격식적 문어)에 대한 선호도를 보이는가?
              \item 이 정도부사들이 공통적으로 가지는, 수식 대상에 의도적으로 모호성을 부여하는 '헤지(hedge)' 기능은 이들의 사용 양상과 어떤 관련이 있는가?
            \end{itemize}
          \item[데이터 요약] SJ-RIKS (Sejong-Research Institute of Korean Studies) 확장판 말뭉치
        \end{description}

  \item \textbf{현영희 (2023).} 신문 주제에 따른 인용명사 구문의 특성 연구, \da{한글}, 84(2), 479-502. \href{https://doi.org/10.22557/HG.2023.6.84.2.479}{https://doi.org/10.22557/HG.2023.6.84.2.479}
  
        \begin{description}[font=\normalfont\bfseries, style=unboxed]
          \item[연구질문] 신문의 주제 영역(경제, 문화, 사회 등)에 따라, 인용명사 구문의 사용 양상과 의미 기능은 어떻게 다른가?
            \begin{itemize}
              \item 각 주제 영역별로 선호되는 인용명사 구문의 유형(생각, 사실, 발화 인용)은 무엇이며, 그 빈도 분포는 어떻게 다른가?
              \item 각 주제 영역에서 특정 인용명사 구문은 화자의 주관성을 강조하거나 내용을 객관화하는 등 어떤 전략적, 인지적 기능을 수행하는가?
              \item 이러한 사용 양상의 차이는 각 주제 영역의 고유한 주제적 특성과 어떤 관련이 있는가?
            \end{itemize}
          \item[데이터 요약] 21세기 세종계획 형태분석 말뭉치의 신문 기사문 중 ‘경제, 문화, 사회’ 3가지 주제 영역
        \end{description}

\end{enumerate}

\section*{패키지 8: 화자에 따른 차이 분석}

\begin{enumerate}[label=\textbf{\arabic*.}] % Makes the number bold with a period

  % --- Add New Papers Here ---
  \item \textbf{문경민 외 (2025).} 한국 중노년층 화자의 의문문 사용 양상: 코퍼스 기반 분석, \da{언어과학}, 32(2), 29-51. \href{http://dx.doi.org/10.14384/kals.2025.32.2.029}{http://dx.doi.org/10.14384/kals.2025.32.2.029}
  
        \begin{description}[font=\normalfont\bfseries, style=unboxed]
          \item[연구질문] 50세 이상 한국어 화자들의 의문문 사용 양상은 노화, 성별, 교육 수준이라는 사회적 요인에 따라 어떻게 달라지는가?
            \begin{itemize}
              \item 나이와 교육 수준은 의문문의 전체 사용 빈도에 각각 어떤 영향을 미치는가?
              \item 성별에 따라 전체 의문문 사용 빈도와, WH-의문문/Y/N-의문문의 선호도에 차이가 있는가?
              \item 노화, 교육, 성별 중 어떤 요인이 의문문 사용에 가장 큰 영향을 미치는가?
              \item 교육적 참여가 노화와 관련된 언어 변화를 상쇄하는 데 도움이 될 수 있는가?
            \end{itemize}
          \item[데이터 요약] AI-Hub 2-18 중노년층 방언 데이터(NIA 2-18 데이터)
        \end{description}

  \item \textbf{이주희, 김소은 (2024).} 메신저 말뭉치의 표기에 나타난 음운 교체 현상: 서울지역의 20 대 남녀를 중심으로, \da{음성음운형태론연구}, 30(3), 343-372. \href{http://dx.doi.org/10.17959/sppm.2024.30.3.343}{http://dx.doi.org/10.17959/sppm.2024.30.3.343}
  
        \begin{description}[font=\normalfont\bfseries, style=unboxed]
          \item[연구질문] 20대 서울 젊은이들의 메신저 대화에서 나타나는 음소 대치(phoneme substitution) 현상은 구어의 음운론적 특징을 어떻게 반영하며, 이러한 양상은 성별에 따라 어떤 차이를 보이는가?
            \begin{itemize}
              \item 메신저 대화에서 나타나는 모음 대치의 빈도와 패턴은 성별에 따라 어떻게 다른가?
              \item 남성과 여성 화자에게서 각각 가장 빈번하게 대치되는 모음은 무엇인가?
              \item 평음과 경음 사이의 자음 대치 선호도는 성별에 따라 어떤 차이를 보이며, 그 이유는 무엇인가?
            \end{itemize}
          \item[데이터 요약] 국립국어원 메신저 말뭉치(버전 2.0) 서울 지역 20대 남녀(친밀도 5 이상)
        \end{description}
        
  \item \textbf{박선우, 박진아, 홍정의 (2015).} SNS 모바일 텍스트의 언어학적 양상: 성별과 연령의 차이를 중심으로, \da{현대문법연구}, 82, 95-120. \href{http://dx.doi.org/10.14342/smog.2015.82.95}{http://dx.doi.org/10.14342/smog.2015.82.95}
  
        \begin{description}[font=\normalfont\bfseries, style=unboxed]
          \item[연구질문] 한국어 SNS 모바일 텍스트의 특징적인 언어 사용 양상은 무엇이며, 이러한 양상은 사용자의 성별과 세대에 따라 어떻게 다른가?
            \begin{itemize}
              \item 비표준적 표기법, 이모티콘, 문장 부호의 사용 빈도는 성별에 따라 어떤 차이를 보이는가?
              \item 텍스트의 길이는 세대에 따라 어떻게 다른가?
              \item 자음만으로 된 초성체(acronyms)와 비표준적 표기법의 사용 빈도는 나이와 어떤 상관관계를 보이는가?
            \end{itemize}
          \item[데이터 요약] ‘페이스북’ 댓글
        \end{description}

\end{enumerate}

\section*{패키지 9: 분석 대상의 확장}

\begin{enumerate}[label=\textbf{\arabic*.}] % Makes the number bold with a period

  \item \textbf{심재형 (2017).} 언어경관 (Linguistic Landscape)과 상권 유동인구 간의 상관관계 : 경기도 안양권(안양, 군포, 의왕) 사례를 중심으로, \da{사회언어학}, 25(1), 85-117. \href{http://dx.doi.org/10.14353/sjk.2017.25.1.04}{http://dx.doi.org/10.14353/sjk.2017.25.1.04}
  
        \begin{description}[font=\normalfont\bfseries, style=unboxed]
          \item[연구질문] 안양권 도시에서, 상업 지구의 유동 인구 규모는 그 지역의 언어 경관(구체적으로, 상업 간판의 영어 사용률)과 어떤 관계가 있는가?
            \begin{itemize}
              \item 유동 인구의 규모와 상업 간판의 영어 사용 비율 사이에는 통계적으로 유의미한 상관관계가 있는가?
              \item 구도심의 쇠퇴와 신흥 상업 지구의 부상과 같은 도시 개발의 양상은 각 지역의 영어 사용률 차이에 어떻게 반영되는가?
              \item 이러한 언어 경관의 차이는 각 상업 지구의 경제적 지위를 어떻게 반영하는가?
            \end{itemize}
          \item[데이터 요약] 안양권 3개 도시(안양, 군포, 의왕)의 8개 주요 역세권 간판 및 유동인구 자료
        \end{description}

  \item \textbf{나은미 (2003).} 人命에 대한 사회언어학적 연구, \da{사회언어학}, 11(1), 95-118. \href{https://scholar-kyobobook-co-kr-ssl.libproxy.snu.ac.kr/article/detail/4030008913271}{교보 스콜라 링크}
  
        \begin{description}[font=\normalfont\bfseries, style=unboxed]
          \item[연구질문] 한국인의 이름에 나타나는 세대별, 성별 차이는 무엇이며, 이러한 변화와 차이에는 어떤 사회적 기준이나 가치가 반영되어 있는가?
            \begin{itemize}
              \item 시대별(1940년대, 60년대, 80년대, 90년대)로, 그리고 성별에 따라 이름에 가장 선호되는 한자는 무엇이며 어떻게 변화했는가?
              \item 이름의 마지막 소리(종성)와 같은 음운론적 특징은 성별에 따라 어떤 차이를 보이는가?
              \item 이러한 선호 한자와 음운론적 특징의 변화는 당대의 사회적 가치나 기준의 변화를 어떻게 반영하는가?
            \end{itemize}
          \item[데이터 요약] 설문지 응답
        \end{description}

  \item \textbf{염수원 (2025).} \sq{ㅟ+ㅓ}의 준말 표기 \sq{ㅕ}의 사용에 관한 양적·질적 탐구: 컴퓨터 매개 문어 말뭉치를 대상으로, \da{사회언어학}, 33(1), 117-143. \href{http://dx.doi.org/10.14353/sjk.2025.33.1.05}{http://dx.doi.org/10.14353/sjk.2025.33.1.05}
  
        \begin{description}[font=\normalfont\bfseries, style=unboxed]
          \item[연구질문] 일상적인 문자 대화에서 비표준 축약형 \sq{ㅕ}는 어떤 빈도로 사용되며, 대화의 순차적 구조 속에서 어떤 상호작용적 기능을 수행하는가?
            \begin{itemize}
              \item 여러 말뭉치에서 축약형 \sq{ㅕ}는 표준형 \sq{ㅟㅓ}와 비교하여 얼마나 자주 사용되며, 화자-청자 간의 친밀도와 같은 관계적 맥락은 그 사용에 어떤 영향을 미치는가?
              \item 대화분석의 관점에서 볼 때, 축약형 \sq{ㅕ}를 사용하는 것은 대화의 순차적 구조 내에서 친화감 형성(affiliation)과 같은 상호작용적 기능을 어떻게 달성하는가?
            \end{itemize}
          \item[데이터 요약] 국립국어원 모두의 말뭉치 내 메신저 말뭉치 (버전 2.0), 비출판물 말뭉치(버전 1.2), 온라인 게시 자료 말뭉치 2022(버전 1.0), 온라인 대화 말뭉치(버전 1.1)
        \end{description}

\end{enumerate}

\section*{패키지 10: 변이와 변화 분석}

\begin{enumerate}[label=\textbf{\arabic*.}] % Makes the number bold with a period

  % --- Add New Papers Here ---
  \item \textbf{김지영, 황선희 (Year).} 말순서 처음에 드러난 담화표지 사용의 세대 차이 연구: 국립국어원 말뭉치 분석을 바탕으로, \da{언어와 정보 사회}, 54, 71-98. \href{http://dx.doi.org/10.29211/soli.2025.54..003}{http://dx.doi.org/10.29211/soli.2025.54..003}
  
        \begin{description}[font=\normalfont\bfseries, style=unboxed]
          \item[연구질문] 10-20대와 50-60대 여성 화자 집단은 발화 시작 위치에서 사용하는 담화 표지(Discourse Markers)의 유형과 기능 면에서 어떤 차이를 보이며, 이러한 차이는 각 세대의 대화 전략을 어떻게 반영하는가?
            \begin{itemize}
              \item 젊은 세대와 나이 든 세대에서 각각 가장 빈번하게 사용되는 담화 표지의 기능적 유형(예: 인지적, 수사적)은 무엇인가?
              \item 젊은 세대는 자신의 생각 과정을 드러내거나 대화의 흐름을 바꾸는 주도적인(dominant) 전략을 더 많이 사용하는가?
              \item 나이 든 세대는 동의나 맞장구를 통해 대화의 흐름을 유지하는 협력적인(cooperative) 전략을 더 많이 사용하는가?
            \end{itemize}
          \item[데이터 요약] 국립국어원 일상 대화 말뭉치 2021’(버전 1.0) 
        \end{description}

  \item \textbf{장교진, 신지영 (2024).} 1990년대와 2010년대 이후 드라마 자료 분석을 통해 살펴본 \{네\}, \{예\} 사용의 변이와 변화, \da{사회언어학}, 32(4), 123-147. \href{http://dx.doi.org/10.14353/sjk.2024.32.4.05}{http://dx.doi.org/10.14353/sjk.2024.32.4.05}
  
        \begin{description}[font=\normalfont\bfseries, style=unboxed]
          \item[연구질문] 1990년대와 2010년대 한국 드라마에서, 화자의 성별과 담화 상황(공적/사적)에 따라 응답 표현 \sq{네}와 \sq{예}의 사용 양상은 어떻게 변화했는가?
            \begin{itemize}
              \item 1990년대 드라마에서, 남성 인물과 여성 인물은 공적 상황과 사적 상황에 따라 \sq{네}와 \sq{예}를 각각 어떻게 다르게 사용했는가?
              \item 2010년대 드라마에서, 남성 인물과 여성 인물은 공적 상황과 사적 상황에 따라 \sq{네}와 \sq{예}를 각각 어떻게 다르게 사용했는가?
              \item  1990년대와 비교했을 때, 2010년대에 나타난 가장 두드러진 변화는 무엇이며, 특히 공적 상황에서의 사용 양상은 어떻게 변했는가?
            \end{itemize}
          \item[데이터 요약] 드라마 8편, 총 40회차
        \end{description}

  \item \textbf{심주희 (2023).}  식당 내 호칭 \sq{저기요}, \sq{여기요}를 대상으로 한 사회언어학적 변이 연구, \da{담화와 인지}, 30(1), 53-78. \href{http://dx.doi.org/10.15718/discog.2023.30.1.53}{http://dx.doi.org/10.15718/discog.2023.30.1.53}
 
        \begin{description}[font=\normalfont\bfseries, style=unboxed]
          \item[연구질문] 식당에서 호출어 \sq{저기요}와 \sq{여기요}의 사용 양상에 나타나는 연령별 차이는, 나이 그 자체 때문인가, 아니면 나이와 관련된 다른 사회적 요인으로 설명될 수 있는가?
            \begin{itemize}
              \item 화자의 나이와 성별은 \sq{저기요}와 \sq{여기요}의 선택에 통계적으로 유의미한 영향을 미치는가?
              \item 나이에 따른 사용 양상의 차이를 다른 요인의 상호작용으로 설명할 수 있는가?
              \item \sq{저기요}와 \sq{여기요}는 어떤 차이를 보이는가?
            \end{itemize}
          \item[데이터 요약] 설문조사
        \end{description}
\end{enumerate}

% \printbibliography[keyword=kobib]
% \printbibliography[notkeyword=kobib,heading=none]


\end{document}