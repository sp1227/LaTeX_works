\documentclass[11pt, aspectratio=169]{beamer}

% --- Core Packages for Modern Documents ---
\usepackage{fontspec}      
\usepackage{unicode-math}
\usepackage{xeCJK}         
\usepackage{graphicx}      
% \usepackage[beamer]{enumitem}
% \usepackage{tcolorbox}
% \tcbuselibrary{minted}
% \usepackage{fontawesome5}
\usepackage{minted}

\xeCJKsetup{CJKspace=true}

% --- Font Setup ---
\setsansfont{Noto Sans KR}
\setmainfont{Noto Serif KR}
\setmonofont{D2Coding}

\setCJKsansfont{Noto Sans KR}
\setCJKmainfont{Noto Serif KR}
\setCJKmonofont{D2Coding}

\setmathfont{Latin Modern Math}

\mode<presentation>
{
  \usetheme{default}      % or try Darmstadt, Madrid, Warsaw, Marburg...
  \usecolortheme{dove} % or try albatross, beaver, crane, dove...
  \usefonttheme{default}  % or try serif, structurebold, ...
  \setbeamertemplate{navigation symbols}{}
  \setbeamertemplate{caption}[numbered]
} 

\setbeamertemplate{sidebar canvas right}[vertical shading][top=gray,bottom=white] 

\AtBeginSection[]{
  \begin{frame}
    \vfill % Vertically center the title
    \centering
    \begin{beamercolorbox}[sep=8pt,center,shadow=true,rounded=true]{title}
      \usebeamerfont{title}\insertsectionhead\par
    \end{beamercolorbox}
    \vfill
  \end{frame}
}

\renewcommand{\arraystretch}{1.3} % Set row height for ALL tables in the document to 1.5x

\definecolor{Highlight}{HTML}{FFF2CC} % A soft yellow

% \setlist[itemize]{itemsep=5pt}

\definecolor{MonokaiBackground}{HTML}{272822}
\definecolor{blockbody}{HTML}{EEEEEE}
\setbeamercolor{block title}{bg=gray, fg=white}
\setbeamercolor{block body}{bg=blockbody, fg=black}
\setbeamertemplate{footline}{
  \hfill % Pushes the content to the right
  \usebeamercolor{page number in head/foot}
  \usebeamerfont{page number in head/foot}
  \insertframenumber{} / \inserttotalframenumber
  \hspace*{2ex} % Adds a little padding from the right edge
}

\setminted{
    style=default, % dracula, native, monokai...
    % everyminted=\color{white},
    linenos,       % Show line numbers
    % bgcolor=MonokaiBackground,
    frame=lines,   % Draw a thin frame around the code
    framesep=2mm,
    xleftmargin=6pt,
    breaklines=true
}

\title[문헌학연습]{문헌학연습: 한국어 자료의 활용}
\subtitle{1주차 문헌의 연구}
\author{김미경}
% \institute{서울대학교}
\date{2025.9.1}


\begin{document}

\begin{frame}
  \titlepage
\end{frame}

% Uncomment these lines for an automatically generated outline.
%\begin{frame}{Outline}
%  \tableofcontents
%\end{frame}

\section{문헌 사용 연구}

\begin{frame}[t]{왜 문헌을 쓰는가}

  \begin{block}{유일한 옵션일 때가 있다}
    \begin{itemize}
      \item 과거의 언어
      \item 용례의 사용량
      \item 용례의 다양성
      \item 화자가 의식하지 못하는 특성
    \end{itemize}
  \end{block}

  \begin{block}{화자를 모을 필요가 없다}
    \begin{itemize}
      \item 가설을 즉각 확인할 수 있다
      \item 시행착오를 몇번이고 거칠 수 있다
      % \item IRB 심의를 거칠 필요가 없다
      % \item 사례비가 들지 않는다
    \end{itemize}    
  \end{block}
\end{frame}

\begin{frame}[t]{문헌을 사용한 언어 연구}
  \begin{block}{경험적 연구(Empirical study)}
    \begin{itemize}
      \item 언어표현 X가 어떻게 쓰이는가?
      \item 언어표현 X가 얼마나 쓰이는가?
      \item 언어표현 X가 무엇과 함께 쓰이는가?
      \item 언어표현 X가 어떤 환경에서 자주 쓰이는가?
      \item 언어표현 X의 출현에 영향을 미치는 요소는 무엇인가?
    \end{itemize}
  \end{block}  
\end{frame}


\section{문헌 관련 분야들}

\subsection{문헌 연구 분야}
\begin{frame}[t]{문헌 연구 분야}
  \begin{block}{고전 연구}
      \begin{tabular}{ccl}
        \textbf{관심사} & \textbf{목표}  & \textbf{학문 분과} \\
        \hline
        사상 & 원전 복원 & 문헌학/고전학 \\
         & & (Philology/Classical studies)\\
        텍스트 & 조어 재구 & 역사비교언어학  \\
         & & (Historical Comparative Linguistics) \\
        매체 & 선본 확립 & 서지학 (Bibliography) \\
      \end{tabular}        
  \end{block}
\end{frame}

\begin{frame}[t]{문헌 연구 분야}
  \begin{block}{모든 기록 연구}
      \begin{tabular}{ccl}
        \textbf{관심사} & \textbf{목표}  & \textbf{학문 분과} \\
        \hline
        매체 & 수집/분류 & 문헌정보학 \\
         & & (Library and Information Science)\\
        텍스트 & 진본 보존 & 기록관리학/기록학  \\
         & & (Archival Science/Studies) \\
      \end{tabular}        
  \end{block}
\end{frame}

\subsection{문헌학 \& 역사언어학}
\begin{frame}[t]{문헌학과 역사언어학의 관계}
  \begin{block}{문헌학의 도구로서의 과거 언어 연구}
    18세기 문헌학 내에 비교 문법 개설
  \end{block}
  % \begin{itemize}[leftmargin=40pt]
  %   \item Comparative philology
  % \end{itemize}

  \begin{block}{언어학의 도구로서의 과거 문헌 연구}
    문헌학은 19세기 역사비교언어학 연구자의 상식
  \end{block}
  % \begin{itemize}[leftmargin=40pt]
  %   \item Comparative linguistics
  % \end{itemize}

  \begin{block}{언어학의 하위 연구 분야}
    \begin{itemize}
      \item 역사비교언어학이 언어학의 하위 분야로 밀려남
      \item 문헌학이 더 이상 상식이 아님
      \item 과거의 언어연구 + 문헌 연구가 언어학과에서 ‘문헌학’으로 통용
    \end{itemize}
  \end{block}
    % \begin{itemize}[leftmargin=40pt]
    %   \item Historical comparative linguistics
    %   \item English Philology, Spanish Philology
    % \end{itemize}    
\end{frame}

\subsection{서지학 \& 역사언어학}
\begin{frame}[t]{서지학과 역사언어학의 관계}
  \begin{block}{언어학의 도구로서 과거 문헌 연구}
    매체의 특성을 통해 텍스트의 생산 시기와 작성자의 계통 판별
  \end{block}

  \begin{block}{한중일}
    ‘서지학’ 활용 (서지학과 문헌학의 분리)
  \end{block}
  
  \begin{block}{서양}
    ‘문헌학’ 활용 (서지학이 문헌학의 일부)
  \end{block}
\end{frame}

\subsection{종합}
\begin{frame}[t]{종합: 왜 문헌을 공부하나?}
  \begin{block}{기록을 용례로 변환}
    문자 해독    
  \end{block}

  \begin{block}{용례의 메타정보 이해}
    \begin{itemize}
          \item 용례의 시기 판별
          \item 용례의 발화자 판별
          \item 용례의 장르 판별
    \end{itemize}
  \end{block}
\end{frame}

\subsection{관련 분야}
\begin{frame}[t]{관련 분야: 이론/역사언어학 연구}
  \begin{block}{분류}
        연구 목적에 따른 언어학의 한 분야
  \end{block}
  \begin{block}{목표}
        언어 표현의 공시적/통시적 특성을 설명함
  \end{block}
    \begin{itemize}
      \item 언어표현 X는 Z라는 성질을 가진다
      \item 언어표현 X의 성질 Z는 시간이 흐르면서 변화했다
    \end{itemize}        
  \begin{block}{문헌과의 관계}
        문헌에서 추출한 용례를 증거로 활용
  \end{block}
\end{frame}

\section{문헌의 확장과 관련분야 확대}

\subsection{‘문헌’ 개념의 확장}
\begin{frame}[t]{‘문헌’ 개념의 확장}
  \begin{block}{{《서지학개론》 서문 중, 한울아카데미, 2004}}
    서지학은 문헌을 대상으로 과학적으로 연구하는 학문이다. 서양에서는 기원전 2세기 고대 그리스에, 동양은 기원전 1세기 전한시대에 비롯되었다. 인류문화의 발전에 따라 문헌의 개념이 변화되어 오늘날은 \textbf{문헌이란 기록된 정보원을 총칭하는 개념}이 되었다. 따라서 문헌이란 기록된  정보를 전달하는 모든 수단 —도서, 잡지, 소책자, 마이크로필름, 오디오 및 비디오테이프, 슬라이드, 점자, 영화필름, 디스켓, 시디롬, 광디스크, 전자책, 데이터베이스 등—을 의미한다.
  \end{block}
\end{frame}

\begin{frame}[t]{‘문헌’ 개념의 확장}
  \begin{block}{새로운 분야의 발전}
      디지털인문학(Digital Humanities)
  \end{block}
  
  \begin{block}{역사언어학의 ‘문헌’ 확대}
    \begin{itemize}
      \item 고문헌
      \item 고문헌 및 20세기 이후 언어의 전산입력자료 (new!)
      \item 전산입력자료로 편찬한 말뭉치 (new!)
    \end{itemize}    
  \end{block}
\end{frame}

\begin{frame}[t]{확장된 문헌의 자료 양태}
  \begin{block}{고문헌}
    20세기 이전 기록물의 원본 및 영인본
  \end{block}

  \begin{block}{전산입력자료}
      \begin{itemize}
        \item 고문헌 및 20세기 이후 출판물의 전산화 자료
        \item 20세기 이후 컴퓨터로 생산된 언어자원 
      \end{itemize}
  \end{block}

  \begin{block}{말뭉치}
      \begin{itemize}
        \item 목적을 가지고 선정된 전산입력자료
        \item 추가 정보가 부착된 전산입력자료
      \end{itemize}
  \end{block}
\end{frame}

\begin{frame}[t]{확장된 문헌의 특성 비교}
  \begin{block}{문헌 비교}
    \begin{center}
      \begin{tabular}{c|c|c}
        \textbf{고문헌} & \textbf{전산입력자료} & \textbf{말뭉치} \\
        \hline
        소수의, 거의 유일한 증거 & 대량의 다양한 증거 & 대표성이 있는 증거 \\
        특이 용례 중심 & 다수 용례 중심 & 분포 중심 \\
        질적 연구 위주 & 질적 및 양적 연구 & 양적 연구 \\
      \end{tabular}
    \end{center}    
  \end{block}
  
\end{frame}

\subsection{확장된 관련 분야들}

\begin{frame}[t]{관련 분야의 확장}
  \begin{block}{과거의 문헌}
    \begin{itemize}
      \item 유물로서의 고문헌을 하나하나 다룸
      \item 용례 하나 하나를 주의깊게 분석
    \end{itemize}
  \end{block}
  \begin{itemize}
    \item 고문헌의 특성과 취급요령: 서지학
    \item 용례의 분석: 이론/역사언어학 (질적) 연구
  \end{itemize}
\end{frame}

\begin{frame}[t]{관련 분야의 확장}
  \begin{block}{확장된 문헌}
    \begin{itemize}
      \item 전산상에 존재하는 대량의 문헌을 동시에 다룸
      \item 용례들의 분포와 빈도가 중요해짐
      \item 연구자가 생각하지 못한 대상을 발견할 수 있게 됨
    \end{itemize}
  \end{block}
  \begin{itemize}
    \item 대량 문헌의 특성과 취급요령: 말뭉치 언어학, 컴퓨터 언어학
    \item 새 패턴 발견: 말뭉치 언어학, 컴퓨터 언어학
    \item 분포와 빈도 세기: (이론/역사언어학) 양적 연구
  \end{itemize}

\end{frame}

\begin{frame}[t]{관련 분야: 양적 연구}
  \begin{block}{분류}
    연구방법론의 일종
  \end{block}

  \begin{block}{목표}
    \begin{itemize}
      \item 언어 표현의 공시적/통시적 특성을 숫자로 표현된 증거로 설명
      \item 숫자의 많고 적음에 의미 부여(항상 전체 수치 필요)
    \end{itemize}
  \end{block}
  
  \begin{block}{문헌과의 관계}
    숫자를 세기 위해 문헌 자료가 필요함
  \end{block}
  % \begin{itemize}
  %   \item 특성: 숫자의 많고 적음에 의미 부여(항상 전체 수치 필요)
  %     \begin{itemize}
  %       \item 십만어절 내에서 100회 출현
  %       \item 백만어절 내에서 1000회 출현
  %     \end{itemize}
  %   \item 관계: 
  % \end{itemize}
\end{frame}

\begin{frame}[t]{관련 분야: 말뭉치 언어학 연구}
  \begin{block}{분류}
    연구 자료에 따른 언어학의 하위 분야
  \end{block}

  \begin{block}{목표}
    실제로 사용된 언어의 패턴을 관찰하고 기술
  \end{block}
  \begin{itemize}
    \item 화자들은 실제로 이런 표현을 사용한다(corpus-based)
    \item 언어 자료에서 예상치 못했던 패턴이 발견되었다(corpus-driven)
  \end{itemize}

  % \begin{block}{주요 관찰 대상}
  %   빈도, 공기어, 문맥
  % \end{block}

  \begin{block}{문헌과의 관계}
    문헌을 분석한다는 것 자체가 분야의 정체성
  \end{block}
\end{frame}

\begin{frame}[t]{관련 분야: 컴퓨터 언어학 연구}
  \begin{block}{분류}
    연구 목적에 따른 응용언어학의 한 분야
  \end{block}

  \begin{block}{목표}
    컴퓨터가 인간의 언어를 이해하고 생성하게 함
  \end{block}
      % \begin{itemize}[leftmargin=40pt]
      %   \item 자동으로 이 말의 형태소/품사를 분석해 줘
      %   \item 자동으로 이 말을 분류/번역/요약해 줘
      %   \item 자동으로 이런 말을 작성해 줘
      % \end{itemize}

  \begin{block}{문헌와의 관계}
    \begin{itemize}
      \item 말뭉치 언어학의 결과물(말뭉치, 분석)을 핵심 재료로 사용
      \item 말뭉치 생산을 주도함
    \end{itemize}
  \end{block}

\end{frame}


\section{문헌학연습}

\subsection{일반적인 목표}

\begin{frame}{문헌학연습의 목표}
  \begin{block}{언어표현 X를 고문헌에서 찾기}
    \begin{itemize}
      \item 과거의 문자 학습
      \item 과거 언어의 문법 학습
      \item 과거 언어 텍스트 읽기 연습(강독)
      \item 과거의 기록물 특성 익히기        
    \end{itemize}    
  \end{block}

  \begin{block}{언어표현 X를 전산입력자료/말뭉치에서 찾기 (new!)}
    \begin{itemize}
      \item 전산입력자료와 말뭉치 파악하기
      \item 말뭉치 편찬 이해하기
      \item 언어표현 추출하기
    \end{itemize}    
  \end{block}
\end{frame}

\subsection{2025-2 목표}

\begin{frame}{2025-2 문헌학연습의 목표}
  \begin{block}{이론/역사언어학자인 나의 바람}
    연구 대상 표현을 관찰하고 싶다
  \end{block}

  \begin{block}{실행상의 장벽}
    \begin{tabular}{ll}
      \textbf{대응} & \textbf{문제} \\
      \hline
      말뭉치에서 찾아보면 되겠지...? & 파일이 너무 많음 \\
      코딩을 배우라고...? & 파이썬이 두려움 \\
      말뭉치 파일 구조를 분석하라고...? & 텍스트 아닌 요소가 너무 많음 \\
      이걸로 무슨 주장을 할 수 있지? & 주장 유형을 모름 \\
    \end{tabular}
  \end{block}
\end{frame}

\begin{frame}[t]{2025-2 문헌학연습의 목표}
  \begin{block}{이 수업의 해결책}
    \begin{itemize}
      \item 전산입력자료/말뭉치에서 연구할 표현 추출하기
      \item 언어 표현의 사용 양상을 토대로 할 수 있는 주장 살펴보기
    \end{itemize}    
  \end{block}

  \begin{block}{목표하지 않는 것}
    \begin{itemize}
      \item 파이썬 전문가
      \item 컴퓨터언어학자
    \end{itemize}      
  \end{block}

\end{frame}

\begin{frame}[t]{전산입력자료 및 말뭉치 분석을 위한 능력}
  \begin{block}{Gries \& Newman (2014) ‘말뭉치 분석 소프트웨어에 필요한 기능’(일부)}
    \begin{itemize}
      \item 여러 개의 파일 열기
      \item 여러 가지 인코딩 대응하기
      \item 단어, 품사, 단어 연쇄 등의 빈도 세기
      \item 정규식을 이용하여 지정한 패턴의 빈도 세기
      \item 검색 결과를 출현 환경과 함께 제시하기
      % \item 검색 결과와 함께 보여줄 출현 환경의 범위를 조절하기
      \item 검색어의 공기어 보여주기
      \item 단어간 관련성의 강도를 계산하기
      % \item n-그램 목록 보여주기
      \item 결과를 저장하고 파일로 내보내기
    \end{itemize}
  \end{block}
\end{frame}

\section{수업 계획}

\begin{frame}[t]{수업 계획}
  \begin{block}{1주차 (9/1)}
    \begin{itemize}
      \item 문헌과 언어학의 관계
      \item 실습: Miniconda, VS Code, Jupyter Notebook 설치
    \end{itemize}    
  \end{block}

  \begin{block}{2주차 (9/8)}
    \begin{itemize}
      \item 선행연구 분석(연구질문 \& 데이터 중심)
      \item 실습: txt 파일 읽어오기
    \end{itemize}    

  % \begin{itemize}[leftmargin=40pt]
  %   \item ‘연세 20세기 한국어 말뭉치’ 검색 결과
  %   \item AI 허브 ‘대규모 구매도서 기반 한국어 말뭉치 데이터’(원천데이터)
  % \end{itemize}

  \end{block}

\end{frame}

\begin{frame}[t]{수업 계획}
  \begin{block}{3주차 (9/15)}
    \begin{itemize}
      \item 선행연구 분석(연구질문 \& 데이터 중심)
      \item 실습: csv, tsv 파일 읽어오기
    \end{itemize}    
  \end{block}

  \begin{block}{4주차 (9/22)}
    \begin{itemize}
      \item 선행연구 분석(연구질문 \& 데이터 중심)
      \item 실습: 데이터프레임 다루기
    \end{itemize}    
  \end{block}
\end{frame}

\begin{frame}[t]{수업 계획}
  \begin{block}{5주차 (9/29)}
    \begin{itemize}
      \item 선행연구 분석(연구질문 \& 데이터 중심)
      \item 실습: docx 읽어오기
      \item 숨겨진 실습: 우리가 자주 쓸 파이썬 코드 이해하기
        % \begin{itemize}
        %   \item ‘21세기 세종계획 결과물’(txt)
        %   \item ‘연세 20세기 한국어 말뭉치’ 검색 결과(xls) 
        %   \item 모두의 말뭉치 ‘국어 역사 말뭉치(1.0)’(xml)
        % \end{itemize}
    \end{itemize}    
  \end{block}

  \begin{block}{6주차 (10/6 추석)}
    보강 대신 자율학습
  \end{block}
\end{frame}

\begin{frame}[t]{수업 계획}
  \begin{block}{7주차 (10/13)}
    \begin{itemize}
      \item 선행연구 분석(연구질문 \& 데이터 중심)
      \item 실습: XML과 그 변종들 읽어오기
        % \begin{itemize}
        %   \item 모두의 말뭉치 ‘일상대화 말뭉치’
        %   \item AI 허브 ‘온라인 구어체 말뭉치 데이터’
        % \end{itemize}
    \end{itemize}    
  \end{block}

  \begin{block}{8주차 (10/20)}
    \begin{itemize}
      \item 개인연구 브레인스토밍
      \item 실습: JSON 읽어오기
    \end{itemize}    
  \end{block}

\end{frame}

\begin{frame}[t]{수업 계획}

  \begin{block}{9주차 (10/27)}
    \begin{itemize}
      \item 개인연구 주제 발표
      \item 실습: 형태소 분석과 품사 부착(koNLPy)
    \end{itemize}    
  \end{block}

  \begin{block}{10주차 (11/3)}
    \begin{itemize}
      \item 실습: 연어 분석
      \item 개인연구 진도보고
    \end{itemize}    
  \end{block}

  \begin{block}{11주차 (11/10)}
    개인연구 중간발표
  \end{block}

\end{frame}

\begin{frame}[t]{수업 계획}
  \begin{block}{12주차 (11/17)}
    \begin{itemize}
      \item 그래프 그리기
      \item 실습: 빈도 분석과 시각화
    \end{itemize}    
  \end{block}

  \begin{block}{13주차 (11/24)}
    \begin{itemize}
      \item 실습: word2vec
      \item 개인연구 진도보고
    \end{itemize}    
  \end{block}

\end{frame}

\begin{frame}[t]{수업 계획}
  \begin{block}{14주차 (12/1)}
    \begin{itemize}
      \item 시기별 데이터를 이용한 변화의 추적
      % \item 데이터 설명하기
      \item 개인연구 진도보고
    \end{itemize}    
  \end{block}

  \begin{block}{15주차 (12/8)}
    개인연구 발표
  \end{block}
\end{frame}

\section{실습}

\begin{frame}[t]{상황 파악}
  \begin{block}{나는...}
    \begin{itemize}
      \item 말뭉치에서 표현을 세서 데이터프레임으로 나타낼 수 있다.
      \item 말뭉치에서 표현을 추출하여 용례만 파일로 저장할 수 있다.
      \item 말뭉치 파일 여럿을 자동으로 읽어올 수 있다.
      \item 파이썬을 다룰 수 있다.
      \item 내 컴퓨터에 파이썬이 설치되어 있다.
    \end{itemize}      
  \end{block}
\end{frame}

\begin{frame}[t]{환경 갖추기: Miniconda 설치}
  \begin{block}{1. 설치 파일 다운로드}
    \begin{itemize}
        \item 접속: \url{https://www.anaconda.com/download/success}
        \item \textbf{Mac 사용자}: 본인 Mac의 칩 종류 확인 후 다운로드
        \begin{itemize}
            \item \textbf{Apple Silicon (M1/M2..):} `Download for Apple Silicon' 클릭
            \item \textbf{Intel:} `Download for Intel' 클릭
        \end{itemize}
        \item \textbf{Windows 사용자}: 제안된 파일 다운로드
    \end{itemize}
  \end{block}  
  \begin{block}{2. 설치 진행}
    \begin{itemize}
        \item \textbf{Mac}: 다운로드한 .pkg 파일 실행
        \item \textbf{Windows}: 다운로드한 .exe 파일 실행 (설치 시 \textbf{`conda init' 관련 옵션 체크} 확인)
    \end{itemize}    
  \end{block}
\end{frame}

\begin{frame}[t, fragile]{환경 갖추기: Miniconda 설치}
  \begin{block}{3. (매우 중요) 터미널/Anaconda Prompt 재시작}
    \begin{itemize}
        \item 설치 완료 후, 열려있는 모든 터미널 창을 닫고 새로 실행
        \item 다음 명령어를 입력해서 설치가 되었는지 확인
    \end{itemize}   
  \end{block}
  \begin{minted}{bash}
python --version
conda --version
  \end{minted}
\end{frame}

\begin{frame}[t,fragile]{환경 갖추기: 파이썬 가상환경 만들기}
  \begin{block}{1. 터미널 실행}
    \begin{itemize}
        \item \textbf{Mac}: `터미널(Terminal)' 앱 실행
        \item \textbf{Windows}: 시작 메뉴에서 \textbf{'Anaconda Prompt'} 검색하여 실행 (\textbf{일반 터미널(cmd)이 아님!})
    \end{itemize}    
  \end{block}
\end{frame}

\begin{frame}[t, fragile]{환경 갖추기: 파이썬 가상환경 만들기}
  \begin{block}{2. 가상환경 생성 및 실행 (아래 명령어 순서대로 입력)}
        \begin{enumerate}
            \item 파이썬 3.13 버전으로 `philology' 가상환경 생성
            \begin{minted}{bash}
conda create -n philology python=3.13 -y
            \end{minted}
            \item 생성한 가상환경 활성화
            \begin{minted}{bash}
conda activate philology
            \end{minted}
            \item 주피터 노트북 연결 프로그램 설치 (프롬프트 맨 앞에 (philology) 확인 후 진행)
            \begin{minted}{bash}
conda install ipykernel -y
            \end{minted}
        \end{enumerate}    
  \end{block}  
\end{frame}

\begin{frame}[t]{환경 갖추기: VS code 설치하기}
  \begin{block}{설치파일 내려받기 및 설치 진행}
    \begin{itemize}
      \item \url{https://code.visualstudio.com}
      \item 제안된 파일 다운로드 및 설치
      \item 설치 마지막 단계에서 `PATH에 추가' 옵션 확인
    \end{itemize}
  \end{block}
\end{frame}

\begin{frame}[t]{환경 갖추기: VS code 설정하기}

  \begin{block}{VS code에서 작업 폴더 열기}
    \begin{itemize}
      \item 원하는 위치에 `언어학실습' 폴더 만들기
      \item VS code 실행 후 `File > Open Folder...' 선택
      \item 방금 만든 `언어학실습' 폴더를 열기
      \item 앞으로 모든 실습 파일은 이 폴더 안에 저장!
    \end{itemize}
  \end{block}

  \begin{block}{환경 갖추기: VS code Extensions 설치하기}
    \begin{itemize}
      \item VS code 왼쪽 메뉴의 `Extensions' 아이콘 클릭
      \item "python" (Microsoft) 검색 후 install
      \item "Jupyter" (Microsoft) 검색 후 install
    \end{itemize}    
  \end{block}
\end{frame}

\begin{frame}[t, fragile]{환경 갖추기: 첫 코드 실행}
  \begin{block}{환경 갖추기: 파이썬 인터프리터 선택해보기}
    \begin{itemize}
      \item 새 파일 > Jupyter Notebook 선택
      \item 새로 열린 파일의 첫 번째 셀에 아래 코드와 같이 입력하고 왼쪽의 실행 버튼 클릭
      \item 커널을 선택하라고 뜨면 목록에서 `philology (Python 3.13.5)' 선택
        \begin{itemize}
          \item (커널 목록에 뜨지 않으면 VS Code를 재시작)
        \end{itemize}
      \item VS code 오른쪽 상단에 버전이 표시되면 성공!
    \end{itemize}
  \end{block}
  \begin{minted}{py}
print("Hello, world!")
  \end{minted}    
\end{frame}

% \begin{frame}[t]{환경 갖추기}
%   \begin{block}{ipykernel 설치하기}
%     \begin{itemize}
%       \item 'Running cells with Python 3.13.6 requires the ipykernel package.' 알림이 뜨면 install 선택
%       \item 설치가 완료되면 다시 셀의 실행 버튼 클릭
%       \item "Hello, World!"가 출력되면 환경 완성!
%     \end{itemize}
%   \end{block}

% \end{frame}

% \section{실습 2}

% \begin{frame}[t]{실습 파일 열기}
%   \begin{block}{eTL에서 실습 파일 내려받기}
%     \begin{itemize}
%       \item data.zip
%       \item week1\_open\_files.ipynb
%     \end{itemize}
%   \end{block}
% \end{frame}
\end{document}