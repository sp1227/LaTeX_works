\documentclass[aspectratio=169]{beamer}
\usepackage{kotex}
\usepackage{graphicx}
\usepackage{tikz}
\usepackage{multicol}
\usepackage{hyperref}
\usepackage{emoji}

% 테마 설정
\usetheme{Madrid}
\usecolortheme{whale}

% 제목 정보
\title{언어의 이해}
\subtitle{1강. 언어와 언어학}
\author{김미경}
\date{}

\begin{document}

% 제목 슬라이드
\frame{\titlepage}

% 목차
\begin{frame}{목차}
\tableofcontents
\end{frame}

\section{언어, 언어학, 언어학자}

\begin{frame}{언어의 두 가지 의미}
\begin{itemize}
\item \textbf{구체적, 개별적 '언어'}
    \begin{itemize}
    \item 한국어, 영어, 중국어, 일본어, 프랑스어 등
    \item 특정 언어 공동체가 사용하는 개별 언어
    \end{itemize}
\vspace{0.5cm}
\item \textbf{추상적 '언어'}
    \begin{itemize}
    \item 말을 이해하고 말로 표현하는 능력
    \item 인간의 보편적 의사소통 능력
    \end{itemize}
\end{itemize}
\end{frame}

\begin{frame}{언어학과 언어학자}
\begin{block}{언어학이란?}
구체적, 개별적 '언어'를 연구해서 추상적 '언어'를 탐구하는 학문
\end{block}

\vspace{0.5cm}

\begin{columns}
\begin{column}{0.5\textwidth}
\textbf{언어학자는?}
\begin{itemize}
\item[\checkmark] 언어를 연구하는 사람
\item[\texttimes] 여러 언어를 구사하는 사람
\end{itemize}
\end{column}
\begin{column}{0.5\textwidth}
\begin{alertblock}{주의}
언어학자 ≠ 다언어 구사자
\end{alertblock}
\end{column}
\end{columns}
\end{frame}

\section{언어를 연구하는 이유}

\begin{frame}{언어 연구의 동기}
\begin{enumerate}
\item \textbf{의사소통}
    \begin{itemize}
    \item 타자와 의사소통하기 위해서 → 개별어학
    \end{itemize}
\vspace{0.3cm}
\item \textbf{비교와 대조}
    \begin{itemize}
    \item 언어 간 차이와 관계 탐구 → 비교/대조 언어학
    \end{itemize}
\vspace{0.3cm}
\item \textbf{인간 고유성 탐구}
    \begin{itemize}
    \item 인간만의 특이성 연구 → 이론언어학
    \end{itemize}
\end{enumerate}
\end{frame}

\section{언어학의 핵심 아이디어}

\begin{frame}{언어학의 토대 (1/3)}
\begin{itemize}
\item 언어는 혼란스러워 보이지만 \textbf{체계}가 있음
\item 이 체계는 \textbf{과학적으로 탐구} 가능
\item 언어 지식은 여러 \textbf{층위}로 구성됨
\item 각 층위마다 요소와 체계가 존재
\end{itemize}
\end{frame}

\begin{frame}{언어학의 토대 (2/3)}
\begin{itemize}
\item 화자들은 \textbf{무한한 표현} 생성 가능
\item 언어는 \textbf{변이}를 가짐
    \begin{itemize}
    \item 사람별, 지역별, 상황별 차이
    \end{itemize}
\item 다양한 언어 체계가 존재하지만 \textbf{보편성}도 존재
\end{itemize}
\end{frame}

\begin{frame}{언어학의 토대 (3/3)}
\begin{itemize}
\item 언어 요소의 \textbf{자의성}
\item 화자들은 규칙을 따르지만 \textbf{의식하지 못함}
\item 언어 습득은 \textbf{타고난 능력}에 기반
\item 언어는 시간에 따라 \textbf{변화}
    \begin{itemize}
    \item 화자의 의지와 무관하게 일어남
    \end{itemize}
\end{itemize}
\end{frame}

\section{언어 지식의 구조}

\begin{frame}{언어 능력과 언어 수행}
\begin{columns}
\begin{column}{0.5\textwidth}
\textbf{언어 능력}\\
(Linguistic Competence)
\begin{itemize}
\item 숨겨진 지식
\item 적절한 표현을 만들어내는 능력
\item 설명할 수 없지만 존재
\end{itemize}
\end{column}
\begin{column}{0.5\textwidth}
\textbf{언어 수행}\\
(Linguistic Performance)
\begin{itemize}
\item 실제 언어 사용
\item 언어 능력의 실현
\item 관찰 가능한 행위
\end{itemize}
\end{column}
\end{columns}

\vspace{0.5cm}
\begin{block}{핵심}
언어 수행은 언어 능력을 연구하기 위한 단서
\end{block}
\end{frame}

\begin{frame}{의사소통의 단계}
\begin{center}
\begin{tikzpicture}[scale=0.8, transform shape]
\node[draw, rounded corners] (1) at (0,0) {정보 구상};
\node[draw, rounded corners] (2) at (0,-1.5) {단어 선택};
\node[draw, rounded corners] (3) at (0,-3) {단어 배치};
\node[draw, rounded corners] (4) at (0,-4.5) {발음 계획};
\node[draw, rounded corners] (5) at (5,0) {음성 실현};
\node[draw, rounded corners] (6) at (5,-1.5) {청각 수신};
\node[draw, rounded corners] (7) at (5,-3) {정보 변환};
\node[draw, rounded corners] (8) at (5,-4.5) {정보 이해};

\draw[->] (1) -- (2);
\draw[->] (2) -- (3);
\draw[->] (3) -- (4);
\draw[->] (4) -| (2.5,-5) |- (5);
\draw[->] (5) -- (6);
\draw[->] (6) -- (7);
\draw[->] (7) -- (8);

\node at (2.5,0.5) {\textbf{화자}};
\node at (2.5,-5.5) {\textbf{청자}};
\end{tikzpicture}
\end{center}
\end{frame}

\section{언어학의 하위 분야}

\begin{frame}{언어학의 하위 분야 (1/2)}
\begin{description}
\item[음성학 (Phonetics)] 
    \begin{itemize}
    \item 말소리의 물리적 특성 연구
    \item 발음과 청취 메커니즘
    \end{itemize}
\item[음운론 (Phonology)]
    \begin{itemize}
    \item 말소리의 체계와 패턴
    \item 가능한/불가능한 발음 연속
    \end{itemize}
\item[형태론 (Morphology)]
    \begin{itemize}
    \item 단어의 내부 구조
    \item 형태소와 단어 형성 규칙
    \end{itemize}
\end{description}
\end{frame}

\begin{frame}{언어학의 하위 분야 (2/2)}
\begin{description}
\item[통사론 (Syntax)]
    \begin{itemize}
    \item 문장의 구조와 어순
    \item 문법적 관계와 규칙
    \end{itemize}
\item[의미론 (Semantics)]
    \begin{itemize}
    \item 언어 표현의 의미
    \item 단어와 문장의 의미 관계
    \end{itemize}
\item[화용론 (Pragmatics)]
    \begin{itemize}
    \item 맥락 속의 언어 사용
    \item 함축과 화행
    \end{itemize}
\end{description}
\end{frame}

\begin{frame}{언어 지식의 구성 요소}
\begin{center}
\begin{tikzpicture}
\draw[thick, fill=blue!20] (0,0) circle (2cm);
\draw[thick, fill=green!20] (3,0) circle (2cm);

\node[align=center] at (0,0) {\textbf{어휘부}\\(Lexicon)};
\node[align=center] at (3,0) {\textbf{규칙}\\(Rules)};

\draw[<->, thick] (0,-2.5) -- (3,-2.5);
\node at (1.5,-3) {\textbf{문법 (Grammar)}};

\node[align=left] at (0,-4) {\small 언어 단위들에\\대한 지식};
\node[align=left] at (3,-4) {\small 결합 패턴에\\대한 지식};
\end{tikzpicture}
\end{center}
\end{frame}

\section{언어와 관련된 지식}

\begin{frame}{쓰기 (Writing)}
\begin{columns}
\begin{column}{0.6\textwidth}
\textbf{특징}
\begin{itemize}
\item 소리 외 물리적 매체 사용
\item 의도적 학습 필요
\item 모든 언어가 문자를 갖지는 않음
\item 수정 가능
\end{itemize}
\end{column}
\begin{column}{0.4\textwidth}
\begin{alertblock}{참고}
약 7,000개 언어 중\\
3,000개는 문자 없음\\
(2019년 기준)
\end{alertblock}
\end{column}
\end{columns}
\end{frame}

\begin{frame}{규범 문법 vs 기술 문법}
\begin{columns}
\begin{column}{0.5\textwidth}
\textbf{규범 문법}\\
(Prescriptive Grammar)
\begin{itemize}
\item 사회적으로 규정된 '올바른' 방법
\item 가치평가적
\item 고정적
\item 예: "십상"(O) vs "쉽상"(X)
\end{itemize}
\end{column}
\begin{column}{0.5\textwidth}
\textbf{기술 문법}\\
(Descriptive Grammar)
\begin{itemize}
\item 실제 사용 패턴 기술
\item 객관적 관찰
\item 변화 반영
\item 예: "일부 화자는 '쉽상'이라 말함"
\end{itemize}
\end{column}
\end{columns}
\end{frame}

\section{언어의 특징}

\begin{frame}{Hockett의 언어 특징 (1/3)}
\begin{enumerate}
\item \textbf{의사소통의 수단}
\item \textbf{의미성 (Semanticity)}
    \begin{itemize}
    \item 언어 요소가 의미/기능을 지님
    \end{itemize}
\item \textbf{기능성 (Pragmatic Function)}
    \begin{itemize}
    \item 타인의 행동과 관계에 영향
    \end{itemize}
\item \textbf{교환성 (Interchangeability)}
    \begin{itemize}
    \item 발신과 수신 모두 가능
    \end{itemize}
\end{enumerate}
\end{frame}

\begin{frame}{Hockett의 언어 특징 (2/3)}
\begin{enumerate}
\setcounter{enumi}{4}
\item \textbf{문화적 전달 (Cultural Transmission)}
    \begin{itemize}
    \item 접촉과 소통을 통한 습득
    \end{itemize}
\item \textbf{자의성 (Arbitrariness)}
    \begin{itemize}
    \item 형식과 의미의 관계가 관습적
    \end{itemize}
\item \textbf{분할성 (Discreteness)}
    \begin{itemize}
    \item 연속적 소리를 분절된 단위로 인식
    \end{itemize}
\end{enumerate}
\end{frame}

\begin{frame}{Hockett의 언어 특징 (3/3)}
\begin{enumerate}
\setcounter{enumi}{7}
\item \textbf{전위성 (Displacement)}
    \begin{itemize}
    \item '지금', '여기'가 아닌 것을 다룸
    \item 존재하지 않는 대상도 가능
    \end{itemize}
\item \textbf{생산성 (Productivity)}
    \begin{itemize}
    \item 새로운 메시지 생성
    \item 무한한 표현 가능
    \end{itemize}
\end{enumerate}

\vspace{0.5cm}
\begin{block}{핵심}
이 9가지 특징이 자연언어를 정의함
\end{block}
\end{frame}

\begin{frame}{언어의 자의성}
\begin{center}
\begin{tikzpicture}
% 개 그림 (간단한 표현)
\draw[thick] (0,0) circle (0.5cm);
\node at (0,0) {\LARGE\emoji{dog}};

% 각 언어의 표현
\node at (-2,-1.5) {한국어: 개};
\node at (2,-1.5) {영어: dog};
\node at (-2,-2.5) {일본어: 犬 (inu)};
\node at (2,-2.5) {프랑스어: chien};

% 화살표
\draw[->] (-0.5,-0.5) -- (-1.5,-1.2);
\draw[->] (0.5,-0.5) -- (1.5,-1.2);
\draw[->] (-0.5,-0.5) -- (-1.5,-2.2);
\draw[->] (0.5,-0.5) -- (1.5,-2.2);
\end{tikzpicture}
\end{center}

\begin{alertblock}{자의성의 의미}
형식(form)과 의미(meaning)는 무관하며, 관행으로 연결됨
\end{alertblock}
\end{frame}

\section{언어 연구의 역사}

\begin{frame}{언어 연구의 역사적 발전}
\begin{itemize}
\item \textbf{고대 그리스}
    \begin{itemize}
    \item 철학의 일부로서 언어 고찰
    \item 플라톤, 아리스토텔레스
    \end{itemize}
\item \textbf{중세 유럽}
    \begin{itemize}
    \item 내성문법, 보편문법 탐구
    \end{itemize}
\item \textbf{19세기}
    \begin{itemize}
    \item 역사비교언어학 발달
    \item 인도유럽어족 연구
    \end{itemize}
\item \textbf{20세기}
    \begin{itemize}
    \item 구조주의 (소쉬르)
    \item 생성문법 (촘스키)
    \end{itemize}
\end{itemize}
\end{frame}

\begin{frame}{소쉬르의 구조주의}
\begin{columns}
\begin{column}{0.5\textwidth}
\textbf{랑그 (Langue)}
\begin{itemize}
\item 사회적 실체
\item 공동체가 공유하는 언어
\item 언어학의 연구 대상
\end{itemize}
\end{column}
\begin{column}{0.5\textwidth}
\textbf{파롤 (Parole)}
\begin{itemize}
\item 개인적 수행
\item 개인이 사용하는 언어
\item 랑그 연구의 자료
\end{itemize}
\end{column}
\end{columns}

\vspace{0.5cm}
\begin{block}{핵심 개념}
\begin{itemize}
\item 계열관계와 통합관계
\item 공시적 연구 vs 통시적 연구
\end{itemize}
\end{block}
\end{frame}

\begin{frame}{촘스키의 생성문법}
\begin{center}
\begin{tikzpicture}[scale=0.9, transform shape]
\node[draw, rounded corners, fill=yellow!20] (ability) at (0,2) {언어 능력};
\node[draw, rounded corners, fill=blue!20] (performance) at (0,0) {언어 수행};
\node[draw, rounded corners, fill=green!20] (cognition) at (4,1) {인지 구조};

\draw[->] (ability) -- (performance);
\draw[<->] (ability) -- (cognition);

\node[align=center] at (0,-1.5) {\small 관찰 가능한\\발화들};
\node[align=center] at (0,3.5) {\small 문법 구조\\생성 능력};
\node[align=center] at (4,2.5) {\small 인간의\\마음 속};
\end{tikzpicture}
\end{center}

\begin{block}{핵심 주장}
언어학의 연구 대상은 언어 능력이며,\\
이는 인간의 인지 구조의 일부
\end{block}
\end{frame}

\begin{frame}{21세기 언어학}
\begin{itemize}
\item \textbf{다양한 이론틀의 공존}
    \begin{itemize}
    \item 생성문법
    \item 인지언어학
    \item 사용 기반 이론
    \end{itemize}
\vspace{0.3cm}
\item \textbf{기술 발전의 활용}
    \begin{itemize}
    \item 대량 언어 자원 연구
    \item 컴퓨터 기반 분석
    \item 실험 언어학
    \end{itemize}
\vspace{0.3cm}
\item \textbf{학제간 연구}
    \begin{itemize}
    \item 신경언어학
    \item 전산언어학
    \item 심리언어학
    \end{itemize}
\end{itemize}
\end{frame}

\section{마무리}

\begin{frame}{핵심 정리}
\begin{itemize}
\item 언어학은 개별 언어를 통해 보편적 언어 능력을 탐구
\item 언어는 체계적이고 과학적으로 연구 가능
\item 언어 지식은 여러 층위로 구성 (음성, 음운, 형태, 통사, 의미, 화용)
\item 언어 능력과 언어 수행의 구분이 중요
\item 언어는 자의적이지만 체계적
\item 언어학은 계속 발전하는 학문
\end{itemize}
\end{frame}

\begin{frame}{참고문헌}
\begin{itemize}
\item Department of Linguistics, The Ohio State University (2022) \textit{Language Files}, 13th ed. Ohio State University Press. Chapter 1.
\item Robins, R. H. (1997) \textit{A Short History of Linguistics}, 4th ed. London and New York: Longman
\end{itemize}
\end{frame}

\begin{frame}
\begin{center}
\Huge{감사합니다}
\end{center}
\end{frame}

\end{document}